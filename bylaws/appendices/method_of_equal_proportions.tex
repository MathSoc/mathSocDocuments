%%METHOD OF EQUAL PROPORTIONS (p)
\section{Method of Equal Proportions}
To allocate the Representative seats on Council, first allocate to each
constituency except the At Large Constituency, a single seat. Then, for each constituency, its priority is $P =
\sqrt{\frac{c}{n(n+1)}}$, where $n$ is the number of seats already allocated to
that constituency, and $c$ is the number of constituents in that constituency.

Once the priorities have been calculated, the constituency with the highest
priority is allocated an additional seat, and its divisor and priority are
recalculated. This process is continued until all 30 seats have been allocated.

If multiple constituencies are tied for the highest priority, then they are all
allocated a seat simultaneously, unless that would bring the total number of
allocated seats above the maximum, in which case the tie will be broken in favor
of the constituents in the constituencies. If there is a further tie, then
previous terms' data shall be consulted in reverse chronological order until a
term is found in which there is no tie. In the exceptionally unlikely event that
the tie remains unbroken, it shall be broken in favor of the first constituency
listed in this document.