\section*{Foreword}

This document is a set of annontations to the governance documents of MathSoc.
The intent behind this document is to provide future MathSoc members with the
insight behind why the rules the way they are, so hopefully they won't have to
change every five years in the future, as has largely been MathSoc's historical
approach. I would like the rules that MathSoc creates to last it for a very long
time. While this will not always be the case, with any luck, hopefully these
annotations will serve future Councillors and volunteers well, and help them
ensure that their changes to MathSoc's rules are well thought out.

As the author of the original form of the current Constitution \& Bylaws, and
many of MathSoc's policies (as well as the \emph{de facto} document maintainer
over the course of my undergraduate degree), I hope that I can bring a valuable
perspective. Oftentimes, I have been the go-to guy for ``why is this rule the way
it is'' and I hope that, with this in place, my departure to greener pastures
won't result in a loss of all that knowledge.

Above all, I hope that this document gets continually updated. Nothing in my
experience, is better at producing a feeling that the rules are outdated than an
inability for people to understand them. If future secretaries, as they update
these documents, also maintain this annotation document, then hopefully the good
wisdom of past Councils will continue, and MathSoc's institutional amnesia will
be lessened.

One last note, for future secretaries: you have one most important job, and that
is \emph{to record the exact text of every motion adopted and incorporate this
into the minutes}. None of your duties is more important than that. This
document contains a consolidated copy of all the governing documents of MathSoc.
But it's not the official source. In a perfect world, you could go back through
all the minutes and reconstruct everything from them, because they are the only
record that gets vetted by Council. That doesn't mean you shouldn't keep the
documents up to date, but they are actually less important. On a related line of
thinking, don't ever be wooed by the notion that only keeping digital copies of
everything is a good idea. Update that minutes binder (although probably the
secretaries have stopped doing that by the time I finish these annotations, and
we're going to lose all the minutes to a server crash).

Alexis Hunt\\
Math undergrad, 2010 to 2014\\
Variously:\\
\indent Representative, MathSoc\\
\indent Vice-President, Academic; MathSoc\\
\indent Secretary, MathSoc\\
\indent Speaker for Council, MathSoc\\
\indent Speaker for Council, Federation of Students\\
\indent Senator \& Governor, University of Waterloo\\
\indent Honouray Lifetime Member, MathSoc\\
Winter 2014

\section*{On \emph{Robert's Rules of Order}}

MathSoc's governing documents are written with \emph{Robert's Rules of Order,
11th ed.} (\emph{RONR}) in mind. \emph{RONR} is prescribed by the Constitution
\& Bylaws to be the parliamentary authority of MathSoc, and its rules prevail
wherever ours our silent. To avoid duplication and the possibility of
inconsistency, rules are rarely repeated.

As a result, it is impossible to properly interpret MathSoc's governing
documents without a propert understanding of how \emph{RONR} interacts
with them. One goal of these annotations is to assist in the interpretation of
the rules in conjuction with \emph{RONR}, but someone looking to make
large rule changes should familiarize themselves with \emph{RONR}.

In particular, chapters XV and XVI, addressing officers, minutes, committees,
boards, and reports are all very relevant to understanding these documents.
Chapter XVIII, Bylaws, is probably also a useful resource.

It is almost certain that, someday, a 12th (and 13th, and 14th, and so forth)
version will be released. When that happens, MathSoc's governance shall
automatically transition to the newest edition, as provided for in the
Constitution \& Bylaws. Bear in mind that any references provided in my
commentary are to the 11th edition.

\textbf{Alexis Hunt, Winter 2014}

\section*{Organization}

There are, broadly, four parts to this document, apart from this
introduction to it. The first two correspond to the rules of MathSoc:
the Constitution \& Bylaws in part 2, and the policies in part 3. The fourth
part is not normative information, but a historical record of significant events
and of decisions that are no longer in force, either because they were not
intended to have long-term effect, or because they expired or were rescinded.
Finally, there is a technical guide to the upkeep of this document and the
rather complicated macro system underlying it.

The meat of the document, the annotated sections, contain up-to-date (as of this
writing, anyway) copies of the documents, and colourful, insightful, and
definitely very interesting commentary. This may include historical background
or references to other provisions which may be helpful when interpreting that
section.

This document is not the official copy of the documents of the Society. The
official copies exist only in the minutes, but are a pain to reconstruct because
you must start at the beginning to do it properly, and the minutes are badly
kept. In case of a discrepancy, the minutes prevail (unless the minutes
themselves are known to be in error, in which case they should be corrected; the
minutes do not prevail over history). In certain cases, liberal interpretation
of a motion may be used to convert it to an ongoing rule, but the minutes should
be the guide to the true form of any such rule or motion.

The secretary, of course, is also responsible for maintaining separate copies of
the documents for easy reading which should be the reference to anyone reading
this.

Additionally, every section includes a signature from myself, and hopefully
future secretaries will include their own signatures on new text, rather than
erasing my own, so that history can be included. I would be upset if future
secretaries didn't remove things that had become inappropriate, however. Useless
history is useless (I wonder which will be lost to the mists of time first: that
meme, or this paragraph...).

\textbf{Alexis Hunt, Winter 2014}
