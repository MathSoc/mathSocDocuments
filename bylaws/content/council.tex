%%COUNCIL (c)
\section{Council}

\subsection{Purpose}
The purpose of Council is to represent the members of the Society 
through the general oversight of the Mathematics Society’s Executive Officers,
to determine member priorities with respect to the expenditure of dues, 
to set advocacy priorities, and to establish long-term goals and vision.

\subsection{Composition}
The Society Council's is comprised of up to thirty-five voting members:
\begin{enumerate}
    \item The Executive Officers, as defined in this document; and
    \item Up to thirty Representatives, allocated as described below.
\end{enumerate}

\noindent Additionally, there are non-voting members of Council:
\begin{enumerate}
    \item The Officers of the Mathematics Society, as defined elsewhere in this
        document;
    \item The Secretary of Council;
    \item The chairs of any committees or boards established by the Society or by
        Council;
    \item Any volunteers or executive assistants appointed by decision of the
        Society;
    \item The editors of \emph{math}{\sf NEWS};
    \item The undergraduate student Senators representing the Faculty and
        undergraduates at-large;
    \item The Dean of the Faculty, or designate;
    \item The representatives of the Faculty to the Students' Council of the
        Federation.
    \item The President of a club under the aegis of MathSoc.
\end{enumerate}

If a person qualifies as both a voting and non-voting member of Council, then
they are a voting member.

\subsection{Representative Allocation}
Each Representative on Council shall represent a single constituency from the
following, determined by reference to the Faculty's undergraduate calendar. For
the purposes of determining if a student is in a program, minors and options are
not counted unless explicitly mentioned. A student may be
counted in more than one constituency. Notwithstanding anything else in this
section, in the Fall and Winter terms, if a student is counted in the First Year constituency, 
then they are counted in no other constituency.

When multiple elections are held simultaneously, a voter in multiple
constituencies may cast a vote for Representatives for any of their constituencies in
any election.

The constituencies are
\begin{enumerate}
  \item First Year, consisting of all math students registered as first-year
    with the University, except for those in the Software Engineering program;
  \item Actuarial Science, consisting of all math students in Actuarial Science
    or Mathematical Finance programs;
  \item Statistics, consisting of all math students in Statistics programs;
  \item Pure Mathematics, Applied Mathematics, and Combinatorics and
    Optimization, consisting of all math students in Pure Mathematics, Applied
    Mathematics, Combinatorics and Optimization, Mathematical Finance, or
    Computational Mathematics programs;
  \item Computer Science, consisting of all math students in programs offered by
    the David R. Cheriton School of Computer Science, as well as all students in
    Computational Mathematics programs;
  \item Business, consisting of all math students in Mathematics/Business
    programs, as well as all students in the Business Administration and
    Computer Science Double Degree program;
  \item Computing and Financial Management, consisting of all math students in
    the Computing and Financial Management program;
  \item Software Engineering, consisting of all math students in the Software
    Engineering program;
  \item Teaching, consisting of all math students in the
    Mathematics/Teaching or Pure Mathematics/Teaching programs;
    and
  \item Mathematical Studies and Other, consisting of all math students in
    Mathematical Studies programs and all math students not counted in one of
    the other constituencies.
  \item At Large, consisting of all math students.  
\end{enumerate}

At the beginning of each Fall term, the Board of Directors shall obtain 
enrollment numbers from the University and determine the allocation of the
thirty Representative seats to constituencies using the method of equal
proportions described in the appendix.

The results of the allocations shall be presented for information at the 
Fall General Meeting.

The allocations determined will be used in the subsequent General Election and
future by-elections, as needed, to elect the student representatives on
Council. For each constituency which is voted for in the General Election, if 
after the General Election no seats are elected in that constituency all but 
one seat will be added to the At Large allocation for that term. Otherwise,
for each constituency which is voted for in the General Election all remaining
seats will be added to the At Large allocation for that term. For greater certainty,
no At Large Representative is elected during the General Election in Fall and Winter terms. 
In Spring terms the seats allocated to the First Year constituency shall be instead allocated
to the At Large constituency for the General Election of that term. 


All Councillors will be elected in accordance with the procedure outlined
elsewhere in this document with the exception of the constituencies below.

\subsubsection{Software Engineering Representatives}
\begin{annotation}
    Seats are awarded to lower year classes first in order to create more
    opportunities for younger students to get involved. Younger students also
    can create more change in the society, and present new ideas. Finally,
    Software Engineering has no first-year representatives, so it is important
    to ensure that this demographic has representation on Council.
    
    \textbf{Tristan Potter, Spring 2018}
\end{annotation}

At the start of each term, the students in each on-stream Software Engineering
class shall elect members of their class to serve as Software Engineering
Representatives for the term, in accordance with the usual process of election
of class representatives in the Faculty of Engineering. The available seats
shall be divided evenly between the classes to elect, with any extra seats going
first to lower-year classes. In the Spring term, seats shall be allocated as if
there is a first-year class, but the first-year seats shall remain vacant.

In the event that there are more on-stream Software Engineering classes than
Software Engineering seats on Council, the lower-year classes shall be given
seats first.

\subsubsection{First Year Representatives}

First Year Representatives shall be elected before October 15th of the same
year, and shall serve until the end of the following Winter term. They may also
resign earlier than this by written submission to the rest of Council. 

First Year Representatives shall be elected in accordance with the election
procedures of the Society in all other aspects.

\subsubsection{At Large Representatives}
At Large Representatives shall be elected during the by-election if seats for this constituency are available. 

\subsection{Terms of Office}
Councillors shall be elected for a term in office that ends in concert with the
school term as defined elsewhere in this document.

Councillors or Councillors-elect can resign by written submission to the rest
of Council.

\subsection{Eligibility Requirements}
In order to run for or serve as a Representative, a member must be in the
constituency of their seat or show proof that they intend to register in their
constituency in their term of office.

If a Councillor or Councillor-elect fails to meet these requirements, then they
do not lose their seat automatically, but may be removed from their seat by a
majority vote of Council.

No member shall occupy more than one voting seat on Council in the same term
simultaneously, but a member may run simultaneously for one Executive seat and
one Representative seat in the same term, and a member in a Representative seat
in a given term may run in a by-election for an Executive seat in that same
term.

\subsection{Duties \& Powers}
Council is responsible to uphold the purposes of the Society and to
ensure that the Society is not abused. It is responsible to hold the Executive
and any other persons involved in Society affairs to account, and the
Representatives are responsible for voicing the concerns and issues of their
constituents and, indeed, to represent them. 

Council has full power to determine student priorities regarding the
general expenditure of members’ dues, to establish student perspective on all
matters in relation to post-secondary education, and to set the Policies of the
Society.

Without restricting the generality of the above, Council is expressly
empowered to:
\begin{enumerate}
    \item Receive regular reports from, and provide regular feedback to the
        Executives;
    \item Prepare the termly budget, in so far as the budget pertains to the 
        expenditure of members’ dues, and
        exempting those portions of the budget restricted by the Board of 
        Directors;
    \item Establish procedures regarding the formation, administration,
        discipline, and disbandment of all Society Clubs and Services;
    \item Delegate representatives to serve on bodies outside the University;
    \item Establish joint committees and councils with the University;
    \item Establish such committees and procedures as required for the conduct
        of its business, including those that define elections and referenda
        processes; and,
    \item Make recommendations to the Board of Directors on any matter
        pertaining to the affairs of the Society.
\end{enumerate}

Council is a fully-constituted assembly in its own right,
and does not report to general meetings, though it is accountable to them and
to the members of the Society at large.

\subsubsection{Limitations on Power}
Council may not make any decision contrary to this document or a
resolution passed at a general meeting or by a referendum, unless the limiting
resolution would present a strong risk of imminent dissolution.

\subsection{Duties of Councillors}

Councillors must, in addition to what is otherwise set out in the by-laws,
the policies, and the procedures of the Society:
\begin{enumerate}
    \item Maintain at least one public office hour per week and shall inform
        their constituents of when they are available; 
    \item Attend Council meetings regularly; 
    \item Attend all general meetings;
    \item Actively engage and consult with students regarding the undergraduate
        math student experience;
    \item Report on any relevant updates and activities within their
        constituency; and,
    \item Report on consultation efforts with their constituents, and the
        results thereof.
\end{enumerate}

If a Councillor misses or is more than one half-hour late for a meeting, that
Councillor shall be deemed delinquent for that meeting. If a Councillor is
delinquent for three or more meetings in a given term, then that Councillor may
be removed from that or any other seat for the remainder of the year by
majority vote with notice.

\subsection{Convocation}
Meetings of Council may be called by any of the following;
\begin{enumerate}
  \item The President;
  \item The Speaker of Council;
  \item The Board of Directors;
  \item Any three voting members of Council, upon petition in writing;
  \item Any twenty-five voting members of the Society, upon petition in writing;
    or
  \item The Dean or their designate.
\end{enumerate}

During the period of classes in each term, Council shall meet no less
than once every three weeks.

\subsection{Notice}
Notice must be provided at least 72 hours in advance of any meeting to every
voting member of Council unless that member explicitly waives their
right to notice before the start of the meeting.

Where notice is required of a motion, notice of that motion must be provided at
least seven days in advance of the meeting at which the motion is to be
considered to every voting member of Council unless that member
explicitly waives their right to notice before the start of the meeting at
which the motion is moved. A full description of the intended motion, such as
the text of a proposed amendment or agreement, must be provided, but the motion
may be amended before or after it is moved as long as the changes remain within
the scope of the motion for which notice was given.

Council may designate a mailing list or similar forum to be the
official notice forum of Council; if this is done, then any notice
sent to that forum is considered to have been sent to every voting member of
Council regardless of whether or not it was received by that member.

\subsection{Quorum}
Two-fifths of all voting members of Council shall constitute a
quorum.

\subsection{Sessions}
For greater certainty, each meeting as called in accordance with this document,
plus its adjournments, constitute a single session of Council.

\subsection{Speaker \& Secretary of Council}
The Speaker of Council and the Secretary of Council shall
be appointed by Council. 

At any time when there is no Speaker, the President shall have the powers and
duties of the Speaker, as appropriate.  At any time when there is no Secretary,
the Secretary of the Society, or designate, shall have the powers and duties of the
Secretary, as appropriate.

If there is no Speaker and/or no Secretary present, than Council may
not proceed to any other business until a Speaker and/or a Secretary is
appointed. 

The term of a Speaker or Secretary of Council shall extend from 
appointment to the end of the current academic term. For greater certainty,
the Speaker and/or Secretary of Council may resign or be replaced by
Council at an earlier time. 

\subsubsection{Duties of the Speaker}
The Speaker has the following duties:
\begin{enumerate}
  \item Serve as the presiding officer of Council;
  \item Work with the Chair of the Board of Directors to interpret this and any
      other governing documents of the Society, subject to appeal to 
      Council, the Board of Directors, or a general meeting;
  \item Arrange for and advertise meetings of Council;
  \item Ensure that all Council members have access to the official
      notice forum, if any; and
  \item Ensure that Council meetings are called regularly.
\end{enumerate}

\subsubsection{Duties of the Secretary}
The Secretary has the following duties:
\begin{enumerate}
  \item Serve as the secretary of Council;
  \item Record attendance of Council meetings, including when a
      member is more than one half-hour late for a meeting and if they sent
      notice of their absence;
  \item Distribute minutes of each Council or General meeting in a
      timely manner after that meeting;
  \item Report to Council when a Councillor is failing to meet the
      requirements of office.
\end{enumerate}

\subsection{Remuneration of Councillors}

Councillors shall not receive monetary remuneration, excluding discounts, 
for serving as such, though they may receive such remuneration for serving 
as officers or employees or in other capacities.
