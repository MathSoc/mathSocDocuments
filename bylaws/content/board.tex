%%Board of Directors (c)
\section{Board of Directors}
\subsection{Purpose}
The Board of Directors shall be responsible for the corporate duties of the
organization, including the management of the legal, financial, and human
resource concerns of the organization as well as setting and reviewing
high-level strategic plans.

\subsection{Composition}
\begin{annotation}
    The structure of the at-large members is meant to place the Board into
    a constant state of transition, which will hopefully aid in knowledge
    sharing and prevent sudden shifts in the attitudes of Board. 

    The size of the Board should allow for quorum to be met (especially with
    remote participation) in any term. 

    The community members allow for more experienced members of the community
    who are interested in the object of the Society to provide expertise to the
    Board and aid in continuity.

    \textbf{Tristan Potter, Winter 2018}
\end{annotation}

The Board of Directors is comprised of up to sixteen (16) voting members:
\begin{enumerate}
    \item The 6 (6) Executive, as defined in this document; 
    \item Two (2) representatives from Council, appointed by Council; 
    \item Two (2) at-large representatives elected in the Fall term;
    \item Two (2) at-large representatives elected in the Spring term;
    \item Two (2) at-large representatives elected in the Winter term; and,
    \item Two (2) community representatives, appointed by the Board of Directors.
\end{enumerate}

\noindent Additionally, the following shall serve as non-voting resource
members:
\begin{enumerate}
  \item The Speaker of Council;
  \item The Secretary of Council;
  \item The Business Manager, as defined elsewhere in this document;
  \item The Vice President, Operations and Finance of the Federation of
      Students, or designate; and
  \item The Dean of the Faculty, or designate.
\end{enumerate}

If a person qualifies as both a voting and non-voting member of the Board of 
Directors, then they are a voting member.

At-large Directors shall be elected at the termly general meeting for the
seat they are being appointed to.

In the event that one or more at-large seats become vacant, the Board of
Directors may appoint a member to fill the vacancy until such a time as a
replacement may be duly elected.

\subsection{Terms of Office}
\begin{annotation}
    As an example, at-large directors appointed in the Winter term will start
    their term on May 1, and will serve until April 30th of the following year.

    \textbf{Tristan Potter, Spring 2018}
\end{annotation}
At-large members of the Board of Directors shall hold office for one (1) year,
commencing at the beginning of the academic term following the one in which
they were elected. For greater clarity, the at-large director appointed in
the Winter term shall start in May of that year, the at-large director 
appointed in the Spring term will start in September of that year, and the
at-large director appointed in the Fall term shall start in January of the 
following year. 

Community members of the Board of Directors shall hold office for up to two (2)
years in one term, commencing and ending at a time defined by the Board of
Directors.

The Councillor representative will serve until the end of the current academic
term, as defined in this document. 

For greater certainty, Directors may serve for an unlimited number of terms, 
consecutive or otherwise. 

If a successor is not found prior to the end a Councillor's term of office, that
member will continue to serve until a replacement is found. 

Directors or Directors-elect can resign by written submission to the rest of
the Board of Directors. 

\subsection{Duties \& Powers}

The Board of Directors shall manage and have full power over the Societies
affairs as required to ensure the long-term survival of the Society.  The Board
of Directors shall make or cause to be made for the Society, in its name, any
kind of contract which the Society may lawfully enter into and, save as
hereinafter provided, generally may exercise all such other powers and do all
such other acts and things as the Society is authorized to exercise and do.

The Board of Directors is expressly further empowered to:
\begin{enumerate}
    \item Approve the budgets of the organization;
    \item Review the finances of the organization, including the annual audit;
    \item Oversee the Executive and all other Officers of the organization in
        the execution of their duties;
    \item Oversee the strategic direction of the organization;
    \item Determine the dates of termly General Meetings;
    \item Establish the staff structure and the human resources procedures of
        the organization;
    \item Establish such committees and procedures as are necessary for the
        effective execution of its duties; and
    \item Determine the signing authorities of the Society. 
\end{enumerate}

\begin{annotation}
    This next clause was present in the section on Council, and it makes sense
    to include it here as well. These bodies are effectively all equivalent
    and accountable to each other.

    \textbf{Tristan Potter, Spring 2018}
\end{annotation}

The Board of Directors is a fully-constituted assembly in its own right, and
does not report to general meetings, though it is accountable to them and to
the members of the Society at large.

Each Director shall have the right to view, upon request, any internal document
or communication relating to the affairs of the Society, subject to any
procedures the Board of Directors may establish in facilitating this process.

\subsubsection{Limitations on Power}
The Board of Directors may not make any decision contrary to this document or a
resolution passed at a general meeting or by a referendum, unless the limiting
resolution would present a strong risk of imminent dissolution.

A Director may not sign a cheque addressed to an organization of which they 
are a member unless the specific expenditure is explicitly authorized by 
the Board of Directors.

\subsection{Convocation}
Meetings of the Board of Directors may be called by any of the following;
\begin{enumerate}
  \item The Chair of the Board of Directors;
  \item The President;
  \item Any three voting members of the Board of Directors, upon petition in
      writing;
  \item The Dean or their designate.
\end{enumerate}

\subsection{Notice}
\begin{annotation}
    Board meetings happen less frequently than Council meetings, and can have
    significant impact on staff, the financial position of the organization,
    and the ability of members to access services. Directors should have enough
    notice to read and collect their thoughts on any materials to be discussed.

    \textbf{Tristan Potter, Spring 2018}
\end{annotation}
    
Notice must be provided at least five (5) business days in advance of any
meeting to every member of the Board of Directors unless that member explicitly
waives their right to notice before the start of the meeting.

A full description of the intended motion, such as the text of a
proposed amendment or agreement, must be provided, but the motion may be
amended before or after it is moved as long as the changes remain within the
scope of the motion for which notice was given.

The Board of Directors may designate a mailing list or similar forum to be the
official notice forum of the Board of Directors; if this is done, then any
notice sent to that forum is considered to have been sent to every member of 
the Board of Directors, regardless of whether or not it was received by that
member.

\subsubsection{Special Meetings}
A special or emergency meeting of the Board may be called for any sufficiently
urgent purpose, by:
\begin{enumerate}
    \item The Chair of the Board of Directors; 
    \item Any one (1) Executive Officer;
    \item Any three (3) members of the Board; or,
    \item A resolution of the Board.
\end{enumerate}

Notice for emergency meetings must be given to all members no less than one
(1) day in advance, and shall include all items of business to be transacted,
including the exact text of any resolution to be voted on.

No special meeting will be held to conduct business that may be postponed until 
a regular meeting.

\subsection{Eligibility Requirements}
In order to serve as an at-large Director, the applicant must be a current voting 
member of the Society.

If a Director or Director-elect fails to meet these requirements, then they
do not lose their seat automatically, but may be removed from their seat by a
majority vote of the Board of Directors.

No member shall occupy more than one voting seat on the Board of Directors in
the same term simultaneously. 

\subsection{Quorum}
One half plus one (1) of all voting members of the Board of Directors shall
constitute a quorum.

\subsection{Sessions}
For greater certainty, each meeting as called in accordance with this document,
plus its adjournments, constitute a single session of the Board of Directors.

\subsection{Duties of Directors}
Directors must, in addition to what is otherwise set out in the by-laws, the
policies, and the procedures of the Society:
\begin{enumerate}
    \item Familiarize themselves with the by-laws, policies, 
        procedures, and any active agreements of the Society;
    \item Attend all Board meetings, as able;
    \item Actively participate in decisions and strategy development.
\end{enumerate}

\subsection{Remuneration of Directors}
Directors shall not receive monetary remuneration, excluding discounts, for
serving as such, though they may receive such remuneration for serving as
officers or employees or in other capacities.
