%% OBJECT (o)
\section{Object}
\begin{annotation}
It's interesting to compare how the society nowadays works in relation to its
constitutional object. Since the Constitution \& Bylaws were entirely revised in
Winter 2011, the departure ought not to be too profound, but in some ways it is.
MathSoc's promotion of activities is frequently limited to the events it olds,
and increasing awareness to the outside community is also lacking---though a few
enterprising students are looking to change that. Representation has been a
strong point, if I do say so myself. I think that MathSoc could also do a lot
more to help the student body accomplish their goals.

On the flip side, a student society with a good object will always find itself
falling short, because its work is never done.

One thing worth noting is that this article is more than just philosophical.
This provides an important basis, both procedurally (see, for example,
\emph{RONR}, 11th. ed., p.~113, ll.~10-14) and politically, for the society to
frame its deliberations and actions. There are always arguments about what
MathSoc should or should not be doing, and this article can be powerfully
persuasive tool in such arguments, when it is applicable.

Also, I cannot understand for the life of me why I didn't format this as a
bulleted list. It's practically unreadable. Somebody should fix that.

\textbf{Alexis Hunt, Winter 2014}

\end{annotation}

The object of the Society shall be to serve, represent, and promote
undergraduate students in the Faculty of Mathematics at the University of
Waterloo by providing services to students; encouraging and co-ordinating
student participation in athletic, cultural, social, academic, and recreational
activities; encouraging cooperation and communication between students;
increasing awareness of the Faculty and of mathematics in general to the outside
community; by representing students to the Faculty and to the University; and by
providing students with the means to help them accomplish their goals both
within and without the University.
