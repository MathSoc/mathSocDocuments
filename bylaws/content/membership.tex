%% MEMBERSHIP (m)
\section{Membership}
\begin{annotation}
The membership section is one of the most confusing, and probably could be
simplified. But it works and nobody really questions it. Still, rewriting this
section would be a good idea.

Basically, we can divide members into three categories:
\begin{enumerate}
  \item Voting members, who are math undergraduate students;
  \item Social but non-voting members, who pay the fee but aren't students; and
  \item Honourary Lifetime Members.
\end{enumerate}
There are some issues here though. So I'll come back to this.

\textbf{Alexis Hunt, Winter 2014}
\end{annotation}

\begin{annotation}
The section was rewritten! 

\textbf{Mary Sybersma, Winter 2022}
\end{annotation}

\subsection{Membership Fee}

The Society shall levy a membership fee to either be collected by the University
as a portion of student fees, or paid directly to the Society. Some members are
exempt from having to pay the membership fee, as defined elsewhere in this
document.

The amount of the MathSoc Fee may be adjusted only through one of the following two mechanisms:
\begin{enumerate}
    \item Once per Fall term, by a resolution of the Board of Directors,
        specifying an adjustment of a percentage equal to or less than the
        increase in the Consumer Price Index for Canada in the previous
        calendar year according to Statistics Canada. This increase is subject
        to ratification at the next General Meeting; or 
    \item Modified or removed by a referendum, or by Council and ratified by a General Meeting, held in accordance with these bylaws.
\end{enumerate}

If a student has arranged fee payment to the satisfaction of the University and
the arranged fees include the fee for a given term, then that student is
considered to have paid the fee for that term, regardless of whether or not the
Society has received the funds.

Council is expressly empowered to regulate through policy or procedure the rights of
persons who do not pay the membership fee, exempting their status as members, and to form
procedures governing fee opt-outs or refunds, where applicable.

\subsection{Voting Membership}
The voting members of the society are the students who meet one or more
of the following criteria:
\begin{enumerate}
  \item A full-time or part-time math student in the current term who has paid the Society membership fee. 
  \item A math co-op student in the current term who was a voting member in the
    previous term.
  \item A voting member in the previous term who is slated to be a full-time or
    part-time math student in the next term and is not currently engaged in
    academic study.
\end{enumerate}

\subsection{Social Membership}
The social members of the society are those students (undergraduate and
graduate), staff, faculty, or alumni at the University who have paid the Society
membership fee, as well as all full-time employees of the Society and all
Honorary Lifetime Members of the Society regardless of whether or not they have
paid the fee.

All MathSoc voting members are also social members. 

\subsection{Honorary Lifetime Members}
The Honorary Lifetime Members of the Society are those persons who have made
exceptionally significant contributions to the Society or towards its goals.
Honorary Lifetime Memberships may be conferred only by a three-quarters majority
vote, conducted by secret ballot, of a general meeting of the Society.

Honorary Lifetime Memberships are valid for the lifetime of the Society and
cannot be revoked. Honorary Lifetime Members cannot have obligations imposed on
them due to their status; if they accept a position within the Society, however,
they are still obligated to fulfill the duties of that position.

\subsection{Rights of Voting Members}
Voting members have the exclusive right to participate in Society
decision-making:
\begin{enumerate}
  \item Vote at general meetings of the Society;
  \item Sign petitions of the Society;
  \item Vote in an election to any position on Council or the
      Executive, or in a referendum of the Society; and
  \item Nominate for, stand as candidate for, or a hold a position on 
      Council, as an Executive Officer, or on the Board of Directors.
  \item Inspect the financial records of the Society and, at their own expense,
    request a professional audit.
\end{enumerate}

\subsection{Rights of Social Members}
Social members have, except where described otherwise in this document, the
right to participate fully in activities in the Society, although this does not
mean that the Society cannot charge a fee for an event or that activities cannot
be limited to some subset of members, provided that all members are given fair
opportunity to be included.

Social members, who are not necessarily voting members, shall additionally have the right to participate in any general meeting
of the Society as non-voting members, and to view any governing documents or
public correspondence of the Society.

