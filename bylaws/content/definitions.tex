%% DEFINITIONS (d)
\section{Definitions}

\begin{annotation}
This section, perhaps, has a few more definitions than are strictly necessary. A
few of the definitions deserver further comment, though.

The ``academic year'' as defined by the University is presently from September
to August. This is different from the fiscal year of all major organizations on
campus, which extends from May to April. This causes some difficulties, from
time to time. In practice, MathSoc takes a year to be whatever is convenient at
the time, because a year is a long time for undergrads.

The word ``term'' is highly overloaded, and this is unfortunate. It can mean
both the term of office of someone in a position, and of course the academic
term about 4 months long. Often you want to use both meanings in the same
sentence. There is no particularly good substitute in other case, because the
word ``tetramester'' hasn't really caught on. Often, the correct meaning needs
to be inferred from context.

The definition of a term here indirectly sets the term of office of Council and
of the Executive. This rarely matters, as good Executive tend to star early and
end late, but it does affect things like who gets invited to Orientation Week
events. It also would affect Council membership, should an emergency arise right
at a term boundary and action need to be taken quickly.

The definitions of ``student'' and ``math student'' are carefully written to be
as inclusive as possible. It seems that the University administration is always
inventing creative ways to blend programs across faculties and institutions,
such as the Double Degree programs with Laurier. Interestingly enough, this
does include students who are out of reach of UW at faraway institutions and who
are on exchange.

The definition of ``first-year student'' is similarly broad, and the intent here
is simply to ensure that when we get a list of first-years from the Registrar's
Office to put into our voter lists, it matches up with our definition of
first-year. The most notable case where a student is first-year but not in first
year is when a student fails enough courses that they fail to advance to 2A, as
your term of study in the Math Faculty is determined entirely by credits.

The final paragraphs of this article, relating to non-voting members and
appointments, are rules that exist just so that this case is covered, and not
because there is any particularly good reason for one version of a rule over
another. In particular, the fact that non-voting members cannot make motions is
rather arbitrary, and the rules for appointments just ensure that someone cannot
show up claiming to be a representative without authority (for my thoughts on
that, however, see the later section on non-voting members of Council!).

\textbf{Alexis Hunt, Winter 2014}
\end{annotation}

In this and any other document of the Society, the following definitions shall
hold unless otherwise specified:
\begin{description}
\item[University]\hfill\\
  The University of Waterloo
\item[Faculty]\hfill\\
  The Faculty of Mathematics of the University
\item[Dean]\hfill\\
  The Dean of the Faculty
\item[Society]\hfill\\
    The Mathematics Society of the University of Waterloo
\item[Federation]\hfill\\
    The Federation of Students, University of Waterloo --- also known colloquially as Feds --- now operating as the Waterloo Undergraduate Student Association (WUSA)
\item[Council]\hfill\\
    The Council of the Society
\item[Board]\hfill\\
    The Board of Directors of the Society
\item[Executive]\hfill\\
    Any of the Executive Officers
\item[member]\hfill\\
  Without further qualification, any social member of the Society
\item[academic year]\hfill\\
  Such a period of time as defined by the Senate of the University
\item[term]\hfill\\
  A period of time, approximately four months long, from the start of classes or
  of Orientation week in one term as defined by the Senate of the University,
  until the day before the start of next term
\item[Winter term]\hfill\\
  The term extending roughly from January to April of any given year
\item[Spring term]\hfill\\
  The term extending roughly from May to August of any given year.
\item[Fall term]\hfill\\
  The term extending roughly from September to December of any given year
\item[student]\hfill\\
  Any person registered as an undergraduate student at the University, including
  persons registered at another post-secondary institution for a joint program
  with the University
\item[math student]\hfill\\
  Any student registered in the Faculty, including in a joint program with
  another faculty or institution
\item[first-year student]\hfill\\
  Any student registered as being in their first year of study in the
  University, regardless of whether or not they are actually in their first year
  of study
\item[VPF]\hfill\\
  The Vice-President, Finance
\item[VPO]\hfill\\
  The Vice-President, Operations
\item[VPI]\hfill\\
  The Vice-President, Internal
\item[VPA]\hfill\\
  The Vice-President, Academic
\item[VPC]\hfill\\
  The Vice-President, Communications
\item[decision of the Society]\hfill\\
  A decision made by Council, a general meeting, a referendum, or any other body
  to which the power to make that decision was duly and legally delegated, or a
  document duly enacted by one of the above
\end{description}

\subsection{Non-voting member}
With regards to any assembly within the Society, a non-voting member has all
rights accorded to regular members except for the right to make motions and
the right to vote.

\subsection{Appointment of Designates}

Where someone is entitled to appoint a designate to serve a function
under this document, such an appointment shall be made by written notice to
the President or to the Speaker of Council, and notice of that appointment
shall be given to Council. More than one person may be so designated, but no
more than one person shall exercise the rights provided to a single person at
any given time.
