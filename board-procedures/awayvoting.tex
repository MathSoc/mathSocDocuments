\section{Away-Voting}
\subsection{Applicability of A-Votes}
This board procedure shall apply to urgent, non-controversial items, normally with a financial commitment of less than \$500 each.

\subsubsection{General}
Any Director entitled to a vote may veto an A-Vote; however, an A-Decision shall be binding, pending ratification at a subsequent board meeting.
Note: Ratification is not withheld unless there is a concern where due process was not followed. 
Ratification must not be withheld solely due to distaste with the outcome of a vote.

\subsubsection{Abuse}
An A-Vote which fails or is forced to be withdrawn shall not be brought again for an A-Vote in the same governing term unless otherwise determined at a meeting with due authority.
An A-Vote may not be held:
\begin{itemize}
    \item  to suspend procedure or bylaw;
    \item  to allow another A-Vote under any circumstance;
    \item  to pass any motion which should be held by secret ballot;
    \item  by secret ballot;
    \item  to pass any motion which contradicts any decision of the society at a general meeting; or
    \item  to pass any motion which contradicts any decision of the society in the last year.
\end{itemize}

\subsection{Due Process for A-Votes}
\subsubsection{Initialization}
An A-Vote may be called by any person(s) entitled to call a meeting of the Board.
An A-Vote is considered initialized when notice has been sent, in the same method by which notice for a meeting of the Board would normally be sent.
As with a motion submitted to a meeting of the Board, an A-Vote requires a Mover and Seconder.

\subsubsection{Veto}
As required by The Procedure, any voting member of Board may veto an A-Vote.
This may be done by sending a written request to the Chair of Board within twenty-four (24) hours of the A-Vote being initialized.

\subsubsection{Timeline}
When initialized, a twenty-four (24) hour question and answer period will begin.
At the end of this period, all questions shall be compiled, responded to by either the Mover or Seconder, and reported to the entirety of the Board.
After responses are submitted, a twelve (12) hour voting period shall begin.

\subsubsection{Voting and Threshold}
An A-Vote shall pass by a three-quarters ($\frac{3}{4}$) majority.
A motion that does not meet the prescribed threshold but would pass with a majority does not fail, but is forced to be withdrawn.
A motion forced to be withdrawn is automatically brought to the next meeting of the Board to be voted upon.