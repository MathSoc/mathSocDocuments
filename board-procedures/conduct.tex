\section{Code of Conduct}
\emph{effective February 18, 2024; replaces November 7, 2023}

\subsection{Purpose}
The Mathematics Society recognizes the tremendous work and dedication our student volunteers put into running events, providing services, and overseeing club activities for the student body. As such, this policy expresses our commitment to foster and maintain an environment that is free of harassment and discrimination, for our students to feel safe. The Society will not tolerate anyone intimidating, humiliating, or sabotaging others in our community. The Society aligns with the Federation's \href{https://wusa.ca/document/harassment-and-discrimination-policy/}{Harassment and Discrimination Policy} and is committed to providing and maintaining a professional environment that is based on respect for the dignity and rights of everyone in the Society.

The Society has outlined a code of conduct to highlight our expectations for all individuals who participate in the Mathematics Society, as well as the steps to handle unacceptable behaviour.

The Code of Conduct does not cover criminal matters. Initiating a Code of Conduct complaint does not exclude other paths, such as going to Special Constable Service. For criminal matters, including threats or acts of physical violence, immediately contact Special Constable Service at ext. 22222 or direct line (519) 888-4911.

\subsection{Expected Behaviour}
All members of the Society shall:
\begin{itemize}
	\item exercise consideration and respect to others in speech and actions;
	\item refrain from demeaning, discriminatory, or harassing behaviour and speech; and
	\item be mindful of surroundings and of others.
\end{itemize}

\subsection{Unacceptable Behaviour}
Unacceptable behaviour includes intimidating, harassing, aggressive, abusive, discriminatory, or derogatory speech or actions in any location covered by the scope of this Code of Conduct, as defined in section 9, \textit{Scope}.

For the purposes of this procedure, terms such as ``harassment'' and ``discrimination'' are clearly defined in the Federation's Harassment and Discrimination Policy.

\subsection{Enforcement}
This Code of Conduct is to be enforced by the following Reporting Officers to the best of their ability:
\begin{itemize}
	\item Executive Officers;
	\item Club Executives;
	\item Event organizers; 
	\item Office Managers;
	\item the Chair of the Board; and
	\item the Business Manager.
\end{itemize}
If the Reporting Officer witnesses unacceptable behaviour, they may impose sanctions in accordance with section 6, \textit{Consequences for Unacceptable Behaviour}, A or B, without first going through a committee, unless they are also involved in the situation in question.

\subsection{Experiencing Unacceptable Behaviour}
If a dangerous situation arises or someone is in distress, immediately contact Special Constable Service at ext. 22222 or direct line (519) 888-4911, or call 911.

If there is a violation of this Code of Conduct, contact the nearest eligible Reporting Officer. Complaints received are to be handled by Reporting Officers who are not involved in the situation or closely related to individual(s) in the complaint.

The Reporting Officer will first interview the subject of the complaint and any witnesses, then consult with the Chair of the Board (or replacement; see below) if further action is required and, if necessary, a committee shall be formed to discuss the complaint. Individuals who are involved or closely related to those involved (as determined by the rest of the committee) may not sit on this committee. The committee shall consist of:
\begin{itemize}
	\item the Reporting Officer;
	\item the Chair of the Board;
	\item two Mathematics Society Executive Officers;
	\item the Business Manager;
	\item the Associate Dean of Undergrad Students of the Faculty, or designate; and
	\item the Executive Director of the Federation, or designate.
\end{itemize}
The Chair of the Board shall chair this committee; if they are ineligible, a suitable replacement shall be appointed by the committee. The committee is appointed on a per-incident basis. The chair of the committee shall be familiar with the Federation's Harassment and Discrimination Policy and Federation's MoU with Societies, and they shall notify the Federation, providing a copy of Appendix A, when a temporary or 1-year ban has been imposed on any individual.

\subsection{Consequences of Unacceptable Behaviour}
Any individual found to be in violation of this Code of Conduct shall receive at least one consequence of the following list, based on what the committee or the Reporting Officer deems appropriate:
\begin{enumerate}[label=\Alph*.]
	\item a verbal warning;
	\item a request to leave the area for the rest of the event or the day, whichever is appropriate;
	\item a temporary ban from all areas governed by this Code of Conduct, as outlined in section 17.9 Scope, lasting the remaining duration of the term. If there is less than one month of classes remaining, this ban is to continue into the subsequent term;
	\item a 1-year ban from all areas governed by this Code of Conduct, in the event of a serious incident or repeated violations; or
	\item filing a formal complaint through the Federation.
\end{enumerate}
If the situation is handled by the Reporting Officer, they may assign either of consequence A or B. If the Reporting Officer deems that a more severe sanction is appropriate, they may contact the Chair of the Board to pursue further action. If the situation is handled by the committee, then the committee may assign any sanction on the above list, as they deem appropriate, with consultation from the Federation.

If an individual who is not permitted to be in an area governed by this Code of Conduct is found in such an area, then any Reporting Officer that is aware of this is to ask the individual to leave. If they refuse, the Reporting Officer is to contact Special Constables and have the individual removed. 

In the event that an individual is banned for a term they have already paid fees for, they may opt out of those fees as per the Mathematics Society refund request requirements.

\subsection{Grievances}
If a sanctioned individual disagrees with the consequences assigned, they may appeal the decision by contacting the Chair of the Board. The Chair will discuss the consequences with the committee, or the Federation depending on the severity of consequences, and determine whether to uphold, change, or remove the consequences assigned. If the Chair of the Board is either personally involved in the complaint or has a close relation with someone involved in the complaint, they are to recuse themselves from both the discussion and the vote on the appeal, and a new chair will be appointed by the committee.

In the event that a 1-year ban has been placed on an individual, and if there are no additional complaints related to the sanctioned individual and no instances of non-compliance with the sanctions from previous three terms, the sanctioned individual may request that their sanction be lifted by the committee (assigned to that incident) which shall not be unreasonably refused. Alternatively, if there are instances of non-compliance or additional complaints about the individual during the 1-year ban, an extension of one term (4 months) will be added to the end-date of the current ban.

\subsection{Confidentiality}
The Mathematics Society recognizes that all members have a right to privacy, and will handle complaints confidentially. As such, proceedings will be kept confidential as described below:
\begin{itemize}
	\item information will only be reported to Special Constable Service when directly required;
	\item information will be kept in the Mathematics Society Business Manager's records and be available to Executive Officers, as outlined in Appendix A; and
	\item possible Reporting Officers will be informed of the sanction on the individual in question as is deemed necessary to enforce the sanction (e.g. at events), but no details of the incident will be disclosed.
\end{itemize}

\subsection{Scope}
This procedure applies to all spaces monitored by the Mathematics Society and the individuals using that space, the aforementioned spaces including but not limited to the main office, Coffee and Donut shop, Comfy Lounge, club rooms, and public spaces that have been booked for MathSoc's use. This policy also applies to any and all communication created for Mathematics Society use, including but not limited to the Mathematics Society digital space and any chat created for office staff and clubs, as well as during any Mathematics Society event.
