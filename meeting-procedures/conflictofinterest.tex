\section{Conflict of Interest}
\emph{effective May 1st, 2019}

\subsection{}


A representative shall be considered to have an actual, perceived, or potential conflict of interest when the opportunity exists for the representative to use the authority, knowledge, or influence of MathSoc Council, or a committee or body thereof, for individual benefit or to preferentially benefit any individual or organization with whom the representative has a familial, personal, fiduciary, or financial relationship.

\subsection{}

All representatives and executives shall be expected to certify, prior to candidacy for election or appointment, that if elected they shall fulfil the responsibilities outlined in this procedure. Failure of a candidate for council or executive to certify the same shall render the individual ineligible for candidacy.

\subsection{}

All representatives will be required to complete and submit a Conflict of Interest Declaration, which will be in congruence with the requirements of this procedure, to the Secretary of Council upon their election or reelection to Council. 

\subsubsection{}

Representatives and executives must submit the form no later than the first council meeting of every term.

\subsubsection{}

For executives and other representatives that are also on the Board of Directors, this required form shall be in addition to and separate from any requirements to submit conflicts of interest to the Board of Directors.

\subsubsection{} 

Representatives elected in a by-election must submit the form no later than the start of the first council meeting that they attend.

\subsubection{} 

In the event that new conflicts of interests arise during the course of a representative or executive’s term, representatives and executives are responsible for ensuring their Conflict of Interest Declaration remains up to date.

\subsection{} 

The Secretary of Council shall ensure that copies of all submitted Conflict of Interest Declarations are available for inspection by any member of MathSoc during any Council meeting. Additionally, the Secretary shall make arrangements upon request for a member of MathSoc unable to attend a Council meeting to instead view the documents (without being able to preserve copies of it) at the MathSoc office.

\subsection{} 

Representatives who have an actual, perceived, or potential conflict of interest, with respect to any matter under consideration by Council, a MathSoc General Meeting, or a committee or body thereof, shall declare the nature and extent of the interest immediately and refrain from taking part in any vote in relation to the matter.

\subsection{} 

When Council, or a committee or body thereof, is of the opinion that a conflict of interest exists that has not been declared, Council, a MathSoc General Meeting, or the committee or body thereof may declare by resolution, carried by two-thirds of its members present at the meeting, that a conflict of interest exists and the representative thus declared to be in conflict refrain from taking part in any discussion or in any vote in relation to the matter.
