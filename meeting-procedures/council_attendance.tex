\section{Council Attendance}
\emph{effective April 4, 2019}

\subsection{Definitions}

\begin{enumerate}

\item A Representative accrues an absence if the Representative does not attend a Council meeting or General meeting.
\item A Representative accrues a partial attendance if the Representative is more than 30 minutes late to a Council meeting or General meeting, or leaves more than 30 minutes prior to the originally scheduled adjournment time as specified in the agenda (notwithstanding any subsequent motions to extend the meeting).
\item An excused absence is defined as an absence or a partial attendance where the Representative in question has provided notice to the Speaker prior to the meeting, or justified adequately to the satisfaction of the Speaker.
\item An unexcused absence is defined as an absence or a partial attendance which does not meet the definition of an excused absence.

\end{enumerate}

\subsection{Attendance of Representatives}

The Secretary of Council is responsible for tracking the attendance of Representatives. 

\subsection{Eligibility for Removal}

When a Representative has accrued three unexcused absences or four total absences during the current Term, they are eligible for removal. The Secretary shall provide notice to the Representative in question of their eligibility for removal. The Secretary will also provide notice of the Representative’s eligibility for removal at the next meeting of Council. 

\subsection{Removal Motions}

If a Representative is eligible for removal, any Representative may make a motion to remove them from their seat. Any vote to remove a Representative must be conducted by secret ballot. Any Representative eligible for removal must not vote on any motion concerning their removal. 

\subsection{Survival of a Removal Motion}

Should a Representative survive a removal motion for absences, they will not be eligible for removal until they have accrued a further unexcused absence, or two total absences during the current term.
