\section{Decisions Between Meetings}
effective March 31, 2020.

\subsection{Overview}
Often striking a balance between reasonable oversight and micromanagement can be difficult.  In
the past, Council meetings have been called to approve several small, non-controversial yet 
urgent items.  This policy shall be used to determine which decisions may be made between 
meetings of Council.  The process through which these decisions are made are determined through applicable procedure(s).

\subsection{Definitions}
\begin{itemize}
    \item \textbf{"Away Vote (A-Vote)"} in the context of this policy shall mean the action of making a decision between meetings of Council. 
    \item \textbf{"Away Decision (A-Decision)"} shall mean the outcome of an A-Vote.
    \item \textbf{Continued Decision (C-Decision)"} shall mean a decision made by the Executive together with the Business Manager, as outlined in this policy. 
    \item \textbf{Term Interim} shall mean the period between the end of one term and the first Council meeting of the next term.
\end{itemize}

\subsection{Applicability of A-Votes}

\subsubsection{General}
This policy shall apply to urgent, non-controversial items, normally with a financial commitment 
of less than \$500 each.

\subsubsection{Veto}
Any Councilor entitled to a vote may veto an A-Vote; however, an A-Decision shall be binding, 
pending ratification at a subsequent council meeting.
\textit{Note: Ratification is not withheld unless there is a concern where due process was not 
followed. Ratification must not be withheld solely due to distaste with the outcome of a vote.}

\subsubsection{Abuse}
An A-Vote which fails or is forced to be withdrawn shall not be brought again for an A-Vote in the same governing term unless otherwise determined at a meeting with due authority.

An A-Vote may not be held:
\begin{enumerate}
    \item to suspend procedure or policy;
    \item to allow another A-Vote under any circumstance;
    \item to pass any motion which should be held by secret ballot;
    \item by secret ballot;
    \item to pass any motion which contradicts any decision of the society at a general meeting; or
    \item to pass any motion which contradicts any decision of the society in the last year.
\end{enumerate}

\subsection{Applicability of C-Decisions}

\subsubsection{General}
C-Decisions may be made only during the Term Interim only to financially support events occurring at the 
beginning of a term, as well as continued operations. C-Decisions shall have a maximum financial commitment 
of the lesser of \$500 and an amount approved by Council in the term prior to the current term.

\subsubsection{Abuse}
A C-Decision may not be used:
\begin{enumerate}
    \item to dictate policy or a stance of the Society; or 
    \item to allow anything which would require more than the financial approval of Council. 
\end{enumerate}

\subsection{Due Process for A-Votes}

\subsubsection{Initialization}
An A-Vote may be called by any person(s) entitled to call a meeting of Council.  An A-Vote is considered 
initialized when notice has been sent, in the same method by which notice for a meeting of Council would 
normally be sent. As with a motion submitted to a meeting of Council, an A-Vote requires a Mover and 
Seconder.

\subsubsection{Veto}
As required by The Policy, any voting member of Council may veto an A-Vote.  This may be done by 
sending a written request to the Speaker of Council within twenty-four (24) hours of the A-Vote being 
initialized.

\subsubsection{Timeline}
When initialized, a twenty-four (24) hour question and answer period will begin.  At the end of this 
period, all questions shall be compiled, responded to by either the Mover or Seconder, and reported to the entirety of Council. After responses are submitted, a twelve (12) hour voting period shall begin.

\subsubsection{Voting and Threshold}
An A-Vote shall pass by a three-quarters (¾) majority.  A motion that does not meet the prescribed 
threshold but would pass with a majority does not fail, but is forced to be withdrawn.  A motion 
forced to be withdrawn is automatically brought to the next meeting of Council to be voted upon.

\subsection{Due Process for C-Decisions}
A C-Decision must be unanimous between all Society Executives in office and the Business Manager.  A 
minimum of three (3) parties are required to make a C-Decision.
