\section{Election Procedure}
\subsection{Elections Committee}
\subsubsection{Composition and Appointment}
The Elections Committee shall be a standing committee of Council. The
Committee shall be comprised of five members. The chair of the Committee is
referred to as the Chief Returning Officer. Members shall serve on the Committee
for the term in which they are appointed.

At the start of each academic term, Council shall appoint the Elections Committee at the first
regular Council meeting. 

Prior to the appointment of the Committee during a term, if urgency demands it,
the President may exercise the powers of the Committee in a way consistent with
its duties.

\subsubsection{Duties}
The Elections Committee is responsible for running the Society's Council
elections. This includes, but is not limited to:
\begin{itemize}
\item Setting the dates of the various phases of elections
\item Advertising elections to the members of the Society
\item Verifying the validity of each nomination
\item Arranging the voting system of the election
\item Making judgments regarding any disputes over or violations of the election
procedure, and imposing penalties for violations
\item Providing a report to Council at the conclusion of each election
\item Recommending amendments to the elections procedure
\end{itemize}

The Elections Committee shall make its decisions in fairly and impartially.
\subsubsection{Power}
The Elections Committee has the power necessary to conduct the elections within
the bounds of the procedure as approved by Council, except that its power to
authorize expenditures is limited to \$750 for each general election, \$150
for each by-election, and \$150 for the First-Year elections at the start of
each Fall term.

The Elections Committee has the power to hold elections and by-elections when
required absent any particular approval from Council. The committee has the power to hold
by-elections in consultation with the Executive Board or the Speaker of Council.

\subsubsection{Conflicts of Interest}
If a member of the Elections Committee is standing as, or intends to stand as,
candidate for election to Council, then that member shall be excluded from
proceedings of the committee that is specifically related to them or to the
election for any of the seats for which they are running. In particular, that
member shall not have a vote, or a right to speak or to be present when such
matters are discussed. Additionally, with regards specifically to the election
to those or seats, the member shall exercise no powers as a member of the
Committee.

If more than two members of the Committee are standing as, or intend to stand
as, candidate for the same seat, this matter shall be reported to Council as
soon as possible by the Committee or by any member thereof.

\subsection{Election Schedule}
An election shall consist of three phases:

\begin{itemize}
\item The nomination period, which shall be between 5 business days;
\item An all-candidates meeting, which shall occur the business day immediately after the end
of the nomination period, at which all qualifying candidates or a proxy of whom must be
present unless excused by the Elections Committee;
\item The campaign period, which shall be between 3 and 7 business days, starting the day
immediately after the all-candidates meeting; and
\item The voting period, which shall be the 2 business days immediately
following the campaigning period.
\end{itemize}

The Elections Committee or Council shall set the entire schedule of an election
before it begins, and the Committee shall advertise the dates.

\subsection{Nominations}
\subsubsection{Contents}
All candidates for election must submit nominations to eligible for election.
A nomination shall set out the following:
\begin{itemize}
\item The name, userid, and contact information of the nominee;
\item A signed statement of the nominee agreeing to follow the elections
procedure;
\item The seat and term for which the nominee is being nominated;
\item The names and userids of a number of Society members endorsing the
nomination, each of whom is eligible to vote in the election to that seat; and
\item The signatures of each endorsee.
\end{itemize}
Five members shall be required to endorse a nomination for a Representative
seat, and ten member shall be required for an executive seat (VPA, President). 

No member of the Elections Committee
shall endorse a nomination, and no one shall endorse their own nomination.

Multiple nominations may be combined and submitted simultaneously.

Notwithstanding the above, a First Year student may replace up to two
endorsement signatures from members who can vote in the constituency for which
they are being nominated with the signatures of twice that many members who can
vote in the First Year constituency.
\subsubsection{Submission}
A nomination shall be submitted directly on the voting platform or through a paper nomination to the Elections Committee. If no member of the Elections Committee is
present when a nomination is submitted to the Society office, an office worker shall
note the date and time of the submission on the nomination, sign it, place the
nomination in an envelope, and ensure that the envelope is delivered to the
Elections Committee.

\subsubsection{Publication}

Once the nomination period has ended, the Elections Committee will inform every Qualifying 
candidate of all other candidates running for the same term and seat.

Once a nomination's validity has been verified, the Elections Committee shall
make the nomination public. If a nomination is deemed invalid, then the
Elections Committee shall immediately inform the nominee as well as every
candidate for that seat. 24 hours after the close of nominations and once all
nominations have been verified, the Elections Committee shall indicate that all
nominations are final and indicate any acclamations.

The names of those endorsing a nomination shall not be published by any party,
but the nomination forms shall be open to inspection by appointment with a
member of the Elections Committee until the approval of the election results.

\subsubsection{Verification}
The Elections Committee shall as soon as possible verify the validity of each
nomination, including the eligibility of the candidate and the validity of each
of the endorsements. If a nomination is deemed invalid due to issues with the
endorsements, then the Elections Committee may give the nominee an additional
business day after the close of the nomination period to secure sufficient
endorsements.

\subsubsection{Withdrawal}
A nomination may be voluntarily withdrawn by a candidate at any time prior to
the beginning of the voting period by written submission directly to a member of
the Elections Committee. Endorsements of a nomination cannot be withdrawn.

\subsubsection{Multiple Nominations}
A candidate may submit multiple nominations when there are elections for more than
one seat occurring. If a candidate is nominated for more than one Executive
position and/or more than one Representative position for the same term, then
the candidate shall, within 24 hours of the close of the nomination period,
withdraw sufficient nominations to bring the candidate into line with the
requirements. If the candidate fails to do so, then at that time, their
nominations shall be deemed withdrawn except, for each term, the first
nomination for an Executive seat and the first nomination for a Representative
seat in the order that the Executives and constituences, respectively, are
listed in the bylaws.

\subsubsection{Acclamation}
If, 24 hours after the close of nominations, there are no more candidates than
seats available in any single election, then all candidates shall be acclaimed
and no vote shall be held for that position. The candidates are still required
to adhere to the remainder of the procedure, in particular to submit an expense
report.

\subsection{Campaigning}
\subsubsection{Basic Requirements}
No candidate shall campaign unless
\begin{itemize}
\item It is during the campaigning period;
\item Their nomination has been received by a member of the Elections Committee;
and
\item They have reviewed the rules of the campaign with an Elections Committee
member.
\end{itemize}

\subsubsection{Spending Limits}
A candidate is limited to spending \$75 on campaign materials for an Executive
election and \$40 on campaign materials for a Representative election. If a
candidate is running for both an Executive and a Representative seat, then
expenses shall be counted against both campaigns unless they are clearly
associated with a specific campaign.

If a candidate spends any money on their campaign, they shall provide the
Elections Committee with a complete account of all campaign expenses by the end
of the voting period. A candidate shall be reimbursed for the full cost of their
expenses provided that they provide receipts or other evidence of the costs, in
accordance with the normal financial procedures of the Society unless penalized by the 
Elections Committee for violations of campaigning rules.

Campaigning materials acquired for free or at a price unavailable to other
candidates shall be reported at a fair market value and shall count towards the
expenses limit at that value. Mate- rials acquired at a reduced price where that
reduced price is, or would have been, available to all candidates may be
accounted for at the reduced price, if the candidate can produce documenta-
tion.

\subsubsection{Campaigning Rules}
When campaigning, candidates shall adhere to all policies of the University and
campaign with high moral standards. Candidates shall take care to ensure that
their methods of campaigning are not offensive or overly annoying.

Posters and other signage shall be limited to the buildings operated by the
Faculty or by St. Jerome's University. In accordance with University Policy 2,
a stamp shall be provided in the Society office for use by candidates, and
candidates are responsible for complying with rules set by Plant Operations.

Campaigning shall be generally permitted to occur without prior approval of the
Elections Committee. Campaign materials should indicate contact information for
the Elections Committee in the event of a complaint.

No person shall remove a candidate's campaign material prior to the close of the
election except under authorization of the candidate or the Elections Committee,
unless such material violates University policy or law.

\subsubsection{Voting Period}
During the voting period, no one shall actively campaign for the election of any
specific candidate. Material posted or distributed prior to the voting period
may remain posted during the voting period. At the conclusion of the voting
period, it is a candidate's responsibility to clean up after their campaign.

During the voting period, a candidate is allowed to answer questions that
require a response that might otherwise be considered campaigning.

\subsubsection{Encouragement to Vote}
Encouraging eligible voters to vote, without supporting or denouncing a specific
candidate or candidates, is not considered campaigning and may done freely by
any party, including a candidate, throughout the entirety of the election,
including the voting period.

The Elections Committee may, within the limits of its own spending
authorizations, authorise the reimbursement of any party for any expenses
incurred in the process of encouraging voters to vote.

\subsection{Voting}
\subsubsection{Means of Voting}
Voting shall be conducted by an electronic poll on either the Society’s website, or the Federation of Students’ website. Voting shall be by means of preferential ballot, in which the voter ranks some or all of the candidates. If permitted by the voting software, a candidate may rank some candidates equally. The candidates shall be listed in random order for each voter.

Each voter shall be permitted to vote in each Executive election, and in one Representative election. If a voter is in multiple constituencies, then the voter must select one in which to cast their vote.

The resolution method shall be decided prior to the voting period and published on the voting website used.

The vote shall be secret.


\subsubsection{Office}
During the voting period, the election shall be prominently advertised in the
Society office, and one computer in the office shall be reserved exclusively for
voting. Instructions shall be made clearly available, and office staff shall
be instructed on how to help members cast their votes.

\subsubsection{Candidate Information}
The Elections Committee shall post to the Society website a brief summary of
each candidate, including a short statement solicited from that candidate. The
Committee shall ensure that this information is made available to voters when
they cast their votes.

\subsubsection{Scrutineers}
If permitted by the voting software, each candidate can appoint one scrutineer to verify and monitor the voting
platform. The scrutineer shall have read-only access to the voting platform
throughout the voting period, but not to the database of votes cast.

\subsubsection{Tiebreaker}
The Elections Committee shall appoint a tiebreaker for each election, who must
be eligible to vote in that election. The tiebreaker shall not cast a normal
vote in that election but instead shall seal and date a strict ranking of all
candidates in an envelope prior to the beginning of the voting period. The
envelope shall be kept in the custody of a member of the Elections Committee
other than the tiebreaker. In the event of a tie, the envelope shall be opened
and the ranking within used to break the tie. The envelope shall not otherwise
be opened.

\subsubsection{Resolution}
The elections shall be resolved through:

\begin{itemize}
    \item Any Single Transferable Vote method permitted by Federation of Students policies or procedures, or
    \item The Schulze STV Method, as described by
Martin Schulze in the 2011 paper \emph{Free Riding and Vote Management under Proportional Representation by the Single Transferable Vote}.
\end{itemize}

The Schulze STV
Method generates an ordered ranking of all possible sets of candidates of size
equal to the number of available seats. For each term, the highest set in the
ranking containing only candidates eligible for that term shall be the winning
set; in the event of multiple such sets, if any are disqualified (per the
definition in the paper) by other such sets ranked equally, then they shall be
discarded. If there are still multiple sets remaining, then the winning set
shall be selected by use of the tiebreaking vote; the sets shall be ordered by
the tiebreaking vote, and then the first set in lexicographic order under the
tiebreaking vote shall be the winning set.

\subsubsection{Publication}
The Elections Committee shall, as soon as the election is completed, publish the
results of the election, and a listing of the ballots received with sufficient
information as to allow for independent verification of the results.

\subsection{Violations}
\subsubsection{Rules}
No candidate shall violate the election rules, or knowingly allow another to do
so. A candidate who fails to report a violation of the rules may be held
personally responsible for the violation.

No candidate shall act in bad faith in any manner concerning the election.

\subsubsection{During Election}
Disputes during the election shall be resolved by the Elections Committee at its
discretion. Complaints are to be submitted in writing to any member of the
Committee, and need not come from a candidate. The Elections Committee may also
take up violations of its own initiative. For violation of the rules, the
Elections Committee may impose a reduction of spending limit, order removal of
campaign material, limit a candidate's means of campaigning, or impose other
penalties as it sees fit, except that it may disqualify candidates only when so
authorized.

A candidate who overspends or fails to submit an expense report shall be
disqualified by the Elections Committee. If the Elections Committee believes
that a candidate should be disqualified for any other reason, then they shall
report the recommendation to Council, and Council can impose the
disqualification. The Elections Committee may impose an alternate penalty to
apply until Council makes its decision and/or if Council chooses not to
disqualify the candidate.

\subsubsection{Publication}
When the Elections Commitee imposes a penalty, they shall inform the penalized
member of the decision and the reasons therefor. They shall publish the penalty
and a brief summary of the reasoning to the Society website, taking care as not
to prejudice voters.  Additionally, when a penalty is imposed on a candidate,
every other candidate for the same seat or seats shall be personally informed
with a brief summary of the penalty.

When the Elections Committee chooses to recommend a disqualification, it shall
inform the candidate, but shall not inform other candidates nor publish the
recommendation until it is reported to Council.

\subsubsection{After Election}
If a candidate feels that he or she has been unfairly treated by the Elections
Committee, or has an issue to raise after the voting period, then the complaint
shall be submitted in writing to the President up to two business days after the
voting period. The President shall raise each such complaint at the next meeting
of Council for Council to decide upon prior to ratifying the election results.

\subsection{Ratification}
The Elections Committee shall report to Council the outcome of each election for
ratification. If Council finds that there has been an irregularity in the
election, it may invalidate the election, in whole or in part, and require that
it be held again from any point. Otherwise, the election results as approved by
Council are final.

In the event that Council is unable to meet to receive the Elections Committee's
report, then the Executive Board may receive the report as well as any appeals
received by the President and act on it them in Council's stead.

\subsection{Referendum Procedure}
Except as noted in this section, a referendum shall be governed by the same
rules as an election, \emph{mutatis mutandis}.

\subsubsection{Petitions}
When a petition for a referendum is received, it shall be verified by the
Elections Committee, which Council shall appoint if necessary. If the petition
is determined to be invalid, the Elections Committee shall return it to the
submitter with an explanation of why it was deemed invalid.

A petition for a refendum on a nonsensical or useless resolution shall not be
accepted. If a petition contains multiple resolutions, some of which are
nonsensical or useless, the petition shall still be accepted for the remaining
resolution or resolutions.

\subsubsection{Schedule}
When a referendum is to be held, the Elections Committee or Council shall set
the schedule. It shall consist of a campaigning period of at least 5 days, and a
voting period of 2 days.

\subsubsection{Campaigning Committees}
For each referendum, there shall be a For Committee and an Against Committee.
Any campaigning that would count against a candidate's expenses limit in an
election may be performed only by members of the appropriate committee, as
approved by that committee. A quorum of a campaigning committee is those members
in attendance.

Any voting member, except a member of the Elections Comitttee, may join or
resign from one of the campaigning committees by submission to a member of the
Elections Committee. Their change in membership is effective as soon as it is
confirmed by a member of the Committee. No one may be a member of both campaign
committees for a referendum. The Elections Committee shall keep a public and
current list of members of the campaign committees, and inform all members of a
campaign committee as soon as that committee's membership changes.

No member of a campaign committee shall engage in any form of campaigning
contrary to the established opinion, for or against, of that committee. Members
of a campaigning committee shall generally be under the same rules as candidates
in an election.

The Elections Committee may call meetings of the campaigning committees, if
necessary.

\subsubsection{Expenses Limit}
The expenses by the members of each committee shall total no more than \$100.
The individual members of the committee shall be reimbursed for their expenses.

\subsubsection{Withholding of Vote}
The Elections Committee may impose as a punishment the withholding of a member's
vote. In such a case, the member shall be permitted to submit a physical vote,
which shall be kept sealed unless the punishment is lifted, in which case it
shall be opened and the vote counted. If it is not used, the vote shall be
destroyed.
