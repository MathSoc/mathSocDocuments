\section{Central Budget}
effective October 24, 2019; replaces July 18, 2019; July 26, 2011; March 6, 2007; November 27, 2006

\subsection{Approval of the Budget}
\begin{enumerate}
\item The Vice-President, Finances shall present a central budget to Council for approval by the end of the first month of each term. The central budget shall include a statement of accounts held by the society.
\item All budget requests shall be submitted to the Vice-President, Finances, no later than the third week of the term.
\item If funding from a previous term was allocated but has not yet been spent, it is forfeit unless mentioned in a budget request for the current term. Council shall not refuse funding on a carry-over item that was allocated in either of the previous two terms, but may choose to reallocate the funds afterwards.
\item First Year Representatives are excused from the first budget meeting of the Fall semester.
\item Once the budget has been approved, changes to the budget may be approved as follows:
\begin{enumerate}
\item In the event that the change is reallocating funds within a section (e.g. Academic Events for one club) without changing the total budget for that section, the written approval of both the Vice-President, Finances and the Business Manager is required. If this approval is not granted, the group requesting the change may appeal to Council for approval.
\item For any other changes, the approval of Council is required.
\end{enumerate}
\end{enumerate}


\subsection{Funding Carry-Over}
\begin{enumerate}
\item If funding from a previous term was allocated but has not yet been spent, it is forfeit unless mentioned in a budget request for the current term. Council shall not refuse funding on a carry-over item that was allocated in either of the previous two terms, but may choose to reallocate the funds afterwards.
\item Funds in the Society's main accounts shall be included in the budget in each term as available funds, except for expenses that have been approved but that have not yet been reimbursed.
\item \$10,000 shall be maintained in the Society's main accounts as a float.
\end{enumerate}

\subsection{Approval of Expenses}
\begin{enumerate}
\item All non-budgeted expenses must be approved by members of the Executive Board or the Society Council as follows.
\begin{enumerate}
\item For expenses of \$100.00 or less, the approval of either the President or Vice-President, Operations, as well as the approval of both the Vice-President, Finances and the Business Manager, is required.
\item If a non-budgeted expense of \$100 or less is not approved in accordance with 20.3.1 a), the group that incurred the expense may appeal to Council for approval.
\end{enumerate}
\item Budgeted expenditures must be approved by either the President or Vice-President, Operations, as well as by both the Vice-President, Finances and the Business Manager. If budgeted expenditures exceed the budgeted amount, then the amount in excess of the budgeted amount is to be treated as a non-budgeted expenditure.
\item The Vice-President, Finances shall report expenditures over the budgeted amount to Council.
\item No expense shall be approved from a previous term's budget after the budget meeting for a term unless the unallocated funding was reported and approved in the central budget in the current term.
\item For the purposes of this section, any written approval of an expense constitutes approval of that expense. Additionally, an individual’s signature on the cheque request form constitutes their approval of the expense.
\end{enumerate}

\subsection{Reimbursement}
\begin{enumerate}
\item Expenses incurred in the course of organizing, planning, and executing items of business for the Society are recoverable as long as the conditions in policies are met.
\item All expense requests must be accompanied by a receipt to be approved.
\item A record of expenditures to be reimbursed are to be submitted to the Executive Board within one week following the event. If exact values for the event are not known, an upper estimate should be provided immediately, and an event summary will be provided with appropriate figures as soon as possible. If this is not complied with the expenditure may not be reimbursed.
\end{enumerate}

\subsection{Income}
\begin{enumerate}
\item Income earned in the course of executing society business shall be counted and recorded on an appropriate income form and submitted to the VPF.
\end{enumerate}

\subsection{Appropriate Use of Funds}
Events that utilize funds collected or managed by the Society, or any other organization directly responsible to the Society to purchase alcoholic beverages must be in compliance with University Policy 21 and Society Policy 30.

\subsection{Joint Events}

When Clubs are desirous of splitting event budgets, they must submit a written request to council, detailing how funds are to be split, for this event to be approved.