\section{Central Budget}
\emph{effective March 6, 2024; replaces October 24, 2019; July 18, 2019; July 26, 2011; March 6, 2007; November 27, 2006}\\

\subsection{Approval of the Budget}
\begin{enumerate}
\item The Vice-President, Finance shall present a central budget to Council for approval by the end of the first month of each term. The central budget shall include a statement of accounts held by the society.
\item All budget requests shall be submitted to the Vice-President, Finance no later than the third week of the term.
\item If funding from a previous term was allocated but has not yet been spent, it is forfeit unless mentioned in a budget request for the current term. Council shall not refuse funding on a carry-over item that was allocated in either of the previous two terms, but may choose to reallocate the funds afterwards.
\item First Year Representatives are excused from the first budget meeting of the Fall semester.
\item Once the budget has been approved, changes to the budget may be approved as follows:
\begin{enumerate}
\item In the event that the change is reallocating funds within a section (e.g. Academic Events for one club) without changing the total budget for that section, the written approval of both the Vice-President, Finance and the Business Manager is required. If this approval is not granted, the group requesting the change may appeal to Council for approval.
\item For any other changes, the approval of Council is required.
\end{enumerate}
\end{enumerate}


\subsection{Funding Carry-Over}
\begin{enumerate}
\item If funding from a previous term was allocated but has not yet been spent, it is forfeit unless mentioned in a budget request for the current term. Council shall not refuse funding on a carry-over item that was allocated in either of the previous two terms, but may choose to reallocate the funds afterwards.
\item Funds in the Society's main accounts shall be included in the budget in each term as available funds, except for expenses that have been approved but that have not yet been reimbursed.
\item \$10,000 shall be maintained in the Society's main accounts as a float.
\end{enumerate}

\subsection{Approval of Expenses}
\begin{enumerate}
\item All non-budgeted expenses must be approved by members of the Executive Board or the Society Council as follows.
\begin{enumerate}
\item For expenses of \$100.00 or less, the approval of either the President or Vice-President, Operations, as well as the approval of both the Vice-President, Finance and the Business Manager, is required.
\item If a non-budgeted expense of \$100 or less is not approved in accordance with 20.3.1 a), the group that incurred the expense may appeal to Council for approval.
\end{enumerate}
\item Budgeted expenditures must be approved by either the President or Vice-President, Operations, as well as by both the Vice-President, Finance and the Business Manager. If budgeted expenditures exceed the budgeted amount, then the amount in excess of the budgeted amount is to be treated as a non-budgeted expenditure.
\item The Vice-President, Finance shall report expenditures over the budgeted amount to Council.
\item No expense shall be approved from a previous term's budget after the budget meeting for a term unless the unallocated funding was reported and approved in the central budget in the current term.
\item For the purposes of this section, any written approval of an expense constitutes approval of that expense. Additionally, an individual's signature on the cheque request form constitutes their approval of the expense.
\end{enumerate}

\subsection{Reimbursement}
\begin{enumerate}
\item Expenses incurred in the course of organizing, planning, and executing items of business for the Society are recoverable as long as the conditions in policies and processes are met.
\item All expense requests must be accompanied by a receipt and proof of payment to be approved, and all other requirements pertaining to the cheque request process as outlined on the Cheque Request Form must be satisfied.
\item A record of expenditures to be reimbursed are to be submitted to the Executive Board within two weeks following the event. Any time extensions on submissions must be approved, in writing, by the Executive Board. If this is not complied with, the expenditure may not be reimbursed.
\item Amounts expensed and reimbursed must fall within Society budget line items which have been approved before the expense was made. For greater certainty, reallocations of budget must be approved before expenses are made unless an explicit exception has been made beforehand by the Vice-President, Finance.
\item Gas expenses incurred on personal vehicles related to approved club budget line-items can be reimbursed at the University standard rate given that all conditions specified in University Policy 31.4.ii.a are met. Gas reimbursement must be submitted along with the corresponding event or purchase as outlined in the cheque request process.
\end{enumerate}

\subsection{Income}
\begin{enumerate}
\item Income earned in the course of executing society business shall be counted and recorded on an appropriate income form and submitted to the VPF.
\end{enumerate}

\subsection{Appropriate Use of Funds}
Events that utilize funds collected or managed by the Society, or any other organization directly responsible to the Society to purchase alcoholic beverages must be in compliance with University Policy 21 and Society Policy 30.

\subsection{Joint Events}
When Clubs are desirous of splitting event budgets, they must submit a written request to council, detailing how funds are to be split, for this event to be approved.

\subsection{External Funding and Sponsorship}
\begin{enumerate}
    \item MathSoc Executives and Clubs may apply or seek funding/sponsorship from external sources to help pay for Capital Improvements, projects, or events for the Society or for a Club.
    \item Once the Executive or Club has received approval for funding from an Endowment or external source, they are required to perform the following steps:
    \begin{enumerate}
        \item Forward a copy of the email funding approval to Vice-President, Finance and cc Business Manager in this process. An invoice will then be generated by WUSA accounting to request the funds.
        \item Confirm with the Vice-President, Finance that funds have been received by MathSoc before spending. 
        \item Submit a cheque request, as per cheque request process, to Vice-President, Finance for reimbursement, noting Endowment Proposal ID or Sponsorship details for internal reconciliation. For assistance with large purchases, contact Vice-President, Finance.
    \end{enumerate}
\end{enumerate}
Note: Funding/sponsorship received includes all applicable taxes. Any amount spent over the allotted amount must be approved by Council before the reimbursement is processed. Any amount left over from the allotted amount will remain in the Club's account for future purchases.

\subsection{Teambuilding Expenses}
Both clubs and MathSoc executives will receive a termly Teambuilding Expense. The Teambuilding Expense of a club including the total expense and relevant policies to be followed are specified in Policy 6.3.1.

The MathSoc Executive Teambuilding Expense allocation is $min((M\cdot F - C - N)\cdot 0.5\%, 300)$ where $M$ is the number of MathSoc fee paying students, $F$ is the MathSoc fee, $C$ is the total CIF deduction and $N$ is the total mathNEWS deduction. This expense may be used at the discretion of the MathSoc Executive team for any purpose related to executive teambuilding, transition, or wellness during the term. All financial policies (notably Policy 20.4, 20.6 and 30) must still be followed when requesting reimbursement. The Executive Teambuilding Expense is a standalone expense and cannot be reallocated into or carried over and will be calculated based on start of term estimates of student fees and deductions.
