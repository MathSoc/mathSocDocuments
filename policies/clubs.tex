\section{Clubs Policy}
\subsection{Overview}
This policy governs all clubs under the aegis of MathSoc. MathSoc establishes
clubs to promote social and academic relations inside programs and/or
departments, inside the Faculty. All clubs are expected to abide by legitimate
decisions of MathSoc and of Council, and MathSoc provides resources to the clubs
in return.

The resources and status provided to a club are not a right, but a privilege.
As such, it is in MathSoc's discretion to continue or discontinue support and/or
recognition.

\subsubsection{Establishment of New Clubs}
Members seeking to establish a new club under MathSoc shall submit the following
to the Vice President, Internal, who shall present the petition to Council:

\begin{itemize}
  \item A contact email for the club;
  \item A constitution for the proposed club;
  \item A current list of executive officers and members for the proposed club;
  \item The signature of a professor or professors who advises for the
	  program(s), sponsoring the establishment of a club for that program.
\end{itemize}

Upon request of a group seeking recognition for a new club for a program that
does not currently have a club, the Vice President, Internal shall assist the
group in advertising and holding an initial meeting (or series of meetings) to
adopt a constitution and elect executives.

Consideration of a petition for a new club at Council requires notice, and shall
be a general order at the meeting when notice has been provided. Council is not
obliged to accept the petition for the new club and may in particular request
additional information from the sponsors.

The Executive may exempt a newly-created club from deadlines outlined in this
policy in the term of their creation, as is appropriate to allow the club to
establish itself.

\subsubsection{Constitution Requirements}
If a club is to have any affiliation with any organization outside of the
University, this must be stated clearly in the constitution of the Club.

Unless the club is affiliated (or seeking affiliation) with another organization
whose rules require a lower (including zero) fee, then a club must charge a
termly membership fee of at least \$2, and this fee must be outlined in its
constitution. If the club is affiliated or seeking affiliation with another
organization with a lower maximum fee, then that maximum must be charged. The
membership fee need not be charged for a new club's first term.

A club must make full membership open to all MathSoc social members and must
restrict it to the same, unless it is affiliated (or seeking affiliation) with
another organization, in which case it may also allow full membership to be
available to members of that organization. A club may have other forms of
membership open to the University community. A club must restrict the privileges
to vote and hold executive positions to full members. Further, two thirds of the executive positions of the club must be occupied by Math students, i.e., voting members of the Society. A club must not practice discrimination in its acceptance of members or of executive.

A club must elect its executive officers for a given term no later than the
third week of that term.

In the event of any conflict between a decision of MathSoc, including a policy,
and the constitution of a club, the decision of MathSoc shall prevail.

\subsubsection{Discipline}
Clubs are expected to behave as upstanding members of the University community
and to contribute to the purposes of the Society. Failure to do so may
constitute grounds for sanction by MathSoc.

A club may be put on probation either by a decision of Council or through the
operation of this policy. If a club is on probation, then MathSoc or the
Federation must proactively monitor the club for violations of policies. At the
meeting after budget meeting of the term after the club was put on probation,
and of every term thereafter as long as the club remains on probation, Council
shall evaluate the probation and consider whether to take any action, including
to lift the probation or to impose sanctions.

All motions to discipline a club or to put a club on probation require notice,
given to the club executive directly as well as to Council, except for at the
meeting mentioned above with respect to clubs that have been on probation since
the start of the term.

Additionally, if a club is on probation then any club executives who continue
to violate any of MathSoc's policies may be removed from their positions by a
two thirds majority vote of Council, and Council may appoint or organize
elections for a replacement. Any individuals who were removed from their club
executive positions this way shall not be eligible to be a club executive for
that club for the duration of the term.

Additionally, as outlined elsewhere in this policy, a club may have its funding
withheld as a consequence of failing to meet a deadline or failing to follow any
decisions of MathSoc, including a policy.

\subsubsection{Disbandment}
A club may be disbanded by Council. Such a decision requires notice, given to
the club executive directly as well as to Council, and a two-thirds vote. A club
will normally only be disbanded for continuing to violate policies or other
decisions of Council while on probation, but the decision to disband a club may
be made by Council in its sole discretion at any time. A decision to disband a
club is not effective until the end of the Council meeting after it is adopted,
and becomes final at that time.

In the event that a club is disbanded, its assets shall be transferred to MathSoc.
However, if the club's constitution says otherwise, then Council must consider
instead transferring some or all of its assets according to the club's
constitution.

\subsection{Operations}
\subsubsection{Events}
All club events shall have event forms filed with the Federation. In accordance
with the Societies Agreement, clubs must receive approval for their events.  The
Vice-President, Internal and the Clubs Director shall assist clubs in filing
event forms and securing approval for events.

All promotional material for events must include MathSoc's logo.

A social event is an event with negligible academic purpose.

\subsubsection{Meetings}
Clubs must hold meetings at least once a term. The President and the Vice
President, Internal shall be permitted to attend all club meetings.

\subsubsection{Records}
A copy of the club constitution, a membership list, and all financial records
must be provided to the Vice President, Internal or Vice President, Finance
upon request. If a club fails to do so within one week of the request, the
club's funding shall be withheld until it provides the requested records.

A Club shall provide a copy of its membership list to MathSoc following the
last event of each term for financial calculations. Any members who join after
this shall not be counted towards the number of club members for the purposes
of calculating the club's social events cap in the next term.

Along with its budget each term, a club shall submit a contact email, a list of
its executive members, and a list of the events it intends to run in that term.

If a club amends its constitution, it must provide an updated copy of the
constitution to the Vice President, Internal. If it has previously provided a
link to an online copy, it must inform the Vice President, Internal that the
constitution has been updated.

\subsubsection{Internal Discipline}
If a club engages in any process of internal discipline, including removing an
executive from office or barring an individual from its office, it shall inform
Council of the situation.

If a club's constitution does not provide for disciplinary measures, then any
club executive may apply to Council to have one of its executive removed from
office or to expel a member of the club.

\subsubsection{Council}
Clubs are expected to have representation at each meeting of Council, and it is
a club's responsibility to ensure that it has a representative at each meeting
so as to be informed of changes and to debate any motions that may arise.
Council is free to consider any business relating to clubs regardless of the
clubs' presence or absence, and a club is not excused from the effects of a
decision because it failed to send a representative.

All club executives are entitled to attend meetings of Council for business that
relates directly to their club or to clubs in general. An executive of a club
who is not a member of Council may, unless the club's president indicates
otherwise, use or share the club president's speaking turns on any business. If
the club president has speaking rights for another reason, they may speak with
those rights in addition to sharing their rights as president.

\subsection{Finances}
\subsubsection{Budget}
Each term, no later than the end of the third week of term, a club shall submit
a budget for the term to MathSoc. The budget must outline the club's expected
spending for the term. The budget must be accompanied by the records specified
elsewhere in this policy.

A club's social events cap is $\max(M(5 + .3F), 250) + R$ where $M$ is the
number of club members who are also MathSoc social members, $F$ is the
club's membership fee, and $R$ is any external revenues being used to fund
social events. All expenses of a club that are approved by Council are
eligible for reimbursement unless otherwise directed by Council.

A club may not spend more on social events each term than its social events cap.
If a club's membership increases over the course of the term, it may request
additional funding for social events up to its new social event cap. If the
club's account balance decreases over the course of the term, this shall not
affect its social event cap.

A club's social event expenses may not exceed its academic event expenses.
Council shall not approve a club's budget that has more spending on social
events than academic events. If a club's actuals repeatedly have less spending
on academic events than social events, then the Vice President, Finance shall
bring the matter to Council.


A club's budget must outline general areas of income and spending. Each social
event must appear as a separate line item. Council may amend the budget as it
sees fit before approving it.

Before the central budget meeting, a club may, for the purpose of a beginning-of-term social event, be allocated an amount at most what the club was allocated for its previous term's beginning-of-term social event, to a maximum of \$500, conditional on the approval of the Vice-President, Finance and the Business Manager. Each such allocation must be reported to Council at the soonest meeting thereafter along with explanations for the approved amounts.

If a club fails to submit its complete budget package on time, its package shall
not be considered at the budget meeting, even if submitted before the meeting,
and the club's funding shall be withheld until its budget is approved by
Council. If a club fails to submit its complete budget package for an entire
term, it shall be put on probation.

Clubs are permitted to spend \$80 (or more, if part of their budget request) per
term for recruitment and elections.

Gift card disbursements are limited to no more than \$50 per person per club event.

\subsubsection{Management of Funds}
A club's funds will be managed by MathSoc.

For large purchases, the Vice President, Finance shall assist the club in
arranging for the purchase to be paid directly rather than by reimbursement.
Small expenses shall normally be reimbursed. All expenses for a club must be
signed off on by one of the club's signing authorities. If a club's funding is
being withheld, no expenses or reimbursements shall be paid for that club.

% Crafted with <3 by a number of accounting students
Reimbursements must be requested in the term that the expense is budgeted. 
An arrangement to submit the request in a future term should be made with
the Vice President, Finance if:
\begin{itemize}
	\item an expense is payable on or after the first day of the final exam period;
	\item a reimbursement will be received after the first day of the final exam period; or
	\item an expense has been made but a reimbursement will be submitted in a 
	future term (due to time or resource constraints).
\end{itemize}

Club account balances will be cleared at the start of each term. Each club shall have access to its own financial records at any time, regarding expenditures and income. 

The club may carry a cash float to collect club membership fees and sales, if applicable. 
The club shall reconcile their cash float with the MathSoc VPF at the end of each term and all money is to be deposited into the Society's bank account by the MathSoc VPF. 

Expenditures and income from and to the club shall be applied directly to the club's account. 
