\section{Borrowing Procedure}
\emph{effective April 4, 2019}

\subsection{Overview}
This procedure governs all lending that happens through the MathSoc office. There are three (3) distinct lending procedures: the standard procedure, the WATSFIC procedure, and the special procedure. The special procedure governs rental of large, expensive items, the WATSFIC procedure governs rental of WATSFIC games in conjunction with the WATSFIC MOU, and the standard procedure governs everything else.

\subsection{Standard Rental Procedure}
With the exception of the items outlined in section 1.4.2 Items With Special Rental Procedure and the games and people outlined in section 1.3 WATSFIC Games, all items that MathSoc rents out are to follow the standard procedure, which is as follows:
\begin{enumerate}
    \item The person renting the item(s) is to give the office staff their WATcard and let us know what items they are renting.
    \item An office worker is to input the rental into the rental tracking system on the computer.
    \item MathSoc will hold the WATcard in the till until the items have been returned, at which point an office worker will return the WATcard.
\end{enumerate}

The Games Director is to maintain a list of games that, if rented out, must be returned by 5:00 PM every Thursday, should the games night take place, to be attached to the games shelf. These games are also not to be checked out on Thursdays, should games night be scheduled for that week. The list should not exceed 20\% of the inventory size. If a game is on that list and is not returned by that point, 1.2.1 Overdue Rentals applies. For all items not on that list, 1.2.1 Overdue Rentals applies after either one (1) month has passed from the date of rental or the end of the term is reached, whichever comes first.

\subsubsection{Overdue Rental Procedure}
If a item that falls under standard rental procedure has been out longer than permitted under 1.1 Standard Rental Procedure, then the following steps are to be taken:
Games Inventory Director (in the case of a board game) or the VPO (in the case of anything else) is to email the person renting the item to inquire as to the status of the item.
If the individual does not respond within a week, then the VPO or President is to escalate to faculty and have faculty email the person renting the item.
If the individual does not respond to the faculty's email within a week, then we are to ask the faculty for a hold to be placed on their account until such time as they return the item. In the event that a hold is not placed at this time, we are to report the failure to return the item to campus police.

If the individual responds to an email, the Games Inventory Director (in the case of a board game) or the VPO (in the case of anything else) is to reply with a date, at least one week after the Games Inventory Director/VPO replies to inform the individual of the date, by which the games must be returned. If the date does not work for the individual, then they may suggest an alternate date which may or may not be accepted at the discretion of the Games Inventory Director or VPO, whichever is responsible for the contact. If the items are not returned by the date set, then the next step in the above list is to be followed (e.g. if an individual who is late returning games responds to the Games Inventory Director and does not return the games by the date decided upon, then we are to escalate to faculty).

If the faculty does not send the email within a week, then the next step in the procedure is to be followed, up to the discretion of the Games Inventory Director (in the case of a board game) or the VPO (in the case of anything else).

\subsubsection{Lost/Damaged Items}
If a borrower loses or damages an item rented under Standard Rental Procedure, they are responsible for buying a new copy of the item and any other items that are combined into the same box and then returning the new copy to MathSoc.

\subsection{WATSFIC Games}
Games owned by WATSFIC are to be marked as such. As per the MOU between WATSFIC and MathSoc, games owned by WATSFIC may be borrowed by WATSFIC members using the following procedure:
\begin{enumerate}
    \item The WATSFIC member will show the office staff their WATSFIC membership.
    \item The checkout will be recorded in the rental tracking system.
    \item The WATSFIC member will be permitted to rent the game without leaving anything in the office.
    \item When the game is returned, that will be marked in the computer.
\end{enumerate}

When WATSFIC members check out games that are not marked as owned by WATSFIC, they are to follow 1.2 Standard Rental Procedure. When non-WATSFIC members check out games owned by WATSFIC, they are to follow 1.2 Standard Rental Procedure.

\subsubsection{Overdue Rentals}
WATSFIC members renting WATSFIC games are covered by the WATSFIC policy on overdue games instead of MathSoc policy.

\subsubsection{Lost/Damaged Items}
WATSFIC members renting WATSFIC games are covered by the WATSFIC policy on lost/damaged games instead of the MathSoc policy.

\subsection{Special Rental Procedure}
A minimum of 48 hours' notice must be provided to rent anything under this procedure.

For items with special rental procedure, the rental procedure is as follows:
\begin{itemize}
    \item The first half of the Equipment Pick Up/Drop Off Receipt form is to be filled out.
    \item If the group renting the item is not affiliated with MathSoc, they are to give MathSoc a \$50 cash deposit.
    \item After the above steps have been completed, the item is to be handed over to the group renting it. MathSoc does not provide transportation of the item outside of MC.
    \item Upon return of the item, the second half of the Equipment Pick Up/Drop Off Receipt form is to be filled out and the deposit is to be returned, except in the following cases:
    \begin{itemize}
        \item Where 1.4.4 Lost/Damaged Items Under the Special Procedure indicates otherwise.
        \item When the item has not been properly cleaned. In this case, if a deposit was not charged, a \$50 cleaning fee will be assessed to the club.
    \end{itemize}
\end{itemize}

Rentals under the special procedure are to last for a maximum of 48 hours.

Organizations not affiliated with MathSoc may rent out equipment under this procedure a maximum of five (5) times per term. This limit can be waived at the discretion of at least two (2) members of the executive team including at least one (1) of the VPO and President.

\subsubsection{Allowed Borrowers}
Only the following groups are allowed to borrow items under Special Rental Procedure: any organizations in group 1 under University Policy 15, any group in class 2,3, or 4 with the approval of the Executive Board, and any group in class 5 with the approval of the University. The borrower must be a member of the faculty of mathematics.

\subsubsection{Items With Special Rental Procedure}
The following items fall under the special rental procedure:
-Projector
-Projector Screen
-Speakers
-Karaoke Machine
-Cotton Candy Machine
-Popcorn Machine

\subsubsection{Overdue Rentals}
If a item that falls under standard rental procedure has been out longer than permitted under 1.4 Special Rental Procedure, then the following steps are to be taken:
\begin{enumerate}
    \item The VPO is to email the person renting the item, as well as contact the club/organization renting the item, to inquire as to the status of the item.
    \item A response is not received within 48 hours, then the VPO or President is to escalate to faculty and have faculty email the person, as well as the club/organization, renting the item.
    \item If a response to the faculty's email is not received within 48 hours, then we are to ask the faculty for a hold to be placed on their account until such time as they return the item. In the event that a hold is not placed at this time, we are to report the failure to return the item to campus police. In either case, the club/organization is to be reported to any parent club/organization for failing to appropriately return the item.
\end{enumerate}

If the individual responds to an email, the VPO is responsible for providing time during which the item can be returned. If the item is not returned within 48 hours of communication, then the next step in the above list is to be followed.

If the faculty does not send the email within 48 hours, then the next step in the procedure is to be followed.

\subsubsection{Lost/Damaged Items}
If a borrower loses or damages an item rented under Special Rental Procedure, the deposit is not to be returned and the group that borrowed the item is responsible for reimbursing MathSoc the full cost of the item.
