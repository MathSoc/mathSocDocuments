\section{Lounge and Hallway Bookings}
\emph{effective April 4, 2019; replaces August 19, 2018}\\

The Mathematics Society follows booking procedures in accordance with 
University Policy 15, Bookings, Use and Reservation of University Facilities
for Activities Not Regularly Timetabled. 

The lounges, MC 3001, MC 3002, and the hallway outside the lounges may be
booked by any organizations in group A, any group in class B,C or D with the approval of the Executive Board, and any group in class E with the approval of the University’s Booking Office, the Dean’s Office, or an Associate Dean of the Faculty and the approval of the Executive Board.


\subsection{Booking Groups}

The following booking groups were outlined in University Policy 15 on 
August 14, 2018.

\begin{description}
\item[A\: Accredited Undergraduate Student Groups] Organized groups of students
    whose membership and constitution have been approved by the Federation of
    Students.

\item[B\: Accredited Graduate Student Groups] Any graduate student group
    approved by the Graduate Student Association.

\item[C\: Accredited University Groups] All departments of the University,
    research groups, Federated \& Affiliated Colleges, and any group, club or
    organization recognized by the University or its Federated \& Affiliated
    Colleges. In the case of the University, the Secretary of the University
    will determine status.

\item[D\: Community Charitable Organizations] Organizations such as the K-W
    Rotary Club, Lions Club, Children's International Summer Village, and
    others as approved by the Secretary of the University.

\item[E\: Non-accredited Student and Off-campus Groups] Organized groups which
    do not fall under A. to D. above.
\end{description}

\subsection{Booking Times}

The C\&D lounge (MC 3002) may be booked between the following hours:
\begin{enumerate}
  \item C\&D Closing to 2:00 am---Monday to Friday;
  \item 8:00 am to 2:00 am---Saturday and Sunday;
  \item Any time at the discretion of the Executive Board.
\end{enumerate}

The Comfy lounge (MC 3001) may be booked between the following hours:

\begin{enumerate}
  \item 6:00 pm to 2:00 am---Monday to Friday;
  \item 8:00 am to 2:00am---Saturday and Sunday;
  \item Any time at the discretion of the Executive Board.
\end{enumerate}

The hallway may be booked between the following hours:

\begin{enumerate}
  \item 9:00 am to 5:00 pm---Monday to Friday;
  \item Any time at the discretion of the Executive Board.
\end{enumerate}

Only one lounge or hallway spot may be booked by the same organization at one
time, except under special circumstances, at the discretion of the Executive
Board.

\subsection{Booking Procedures}

A booking request should be completed in its entirety by a representative of
the organization making the booking. The request must be made at least 48 hours
before the start of the event.

The booking space shall be cleaned after use. The organization completing the
booking shall be held responsible for any damage, caused directly or indirectly
by their occupation of the space. If the space is not cleaned or damage occurs,
the organization will be charged for any costs incurred. Additionally, the
booking rights of the organization, at the discretion of the Executive Board,
may be suspended for up to four (4) months following the incident.

For organizations that are not affiliated with MathSoc, the Comfy Lounge or the
Math CnD may be booked 3 times per term and the 3\textsuperscript{rd} floor
hallway may be booked 5 times per term. These limits may be waived at the
discretion of the Vice President, Operations and one other member of the
Executive Board.

If the booking organization is not present in the space 30 minutes after the
start of the booking time, their booking for the day shall be considered
withdrawn and a warning will be issued. If the same organization is late again
during that term, their right to book space, at the discretion of the Executive
Board, may be suspended for up to four (4) months following the incident.

\section{Equipment Bookings}
\emph{effective August xx, 2018; replaces unknown}

Any piece of equipment in the following list, or any other available equipment,
may be booked by any organizations in group A under University Policy 15, any
group in class B, C or D with the approval of the Executive Board, and any
group in class E with the approval of Council.

\begin{itemize}
  \item Projector (and Projector Screen)
  \item Speakers
  \item Karaoke Machine
  \item Popcorn Machine
  \item Cotton Candy Machine
  \item Canon Camera
\end{itemize}

The equipment borrower must be a student in the Faculty of Mathematics.

\subsection{Booking Procedures}

A booking request should be completed in its entirety by a
representative of the organization making the booking. Any organization not
affiliated with MathSoc will also be required to entrust the Executive Board
with a \$50 security deposit. The request and the deposit drop off must be made
at least 48 hours before the start of the event.

This security deposit will be returned to the borrower upon return of all the
equipment except in the following cases:

\begin{enumerate}
  \item If there is damage to any equipment that would cause it to not function
    properly. Should this be the case, the booking organization will be
    responsible for reimbursing MathSoc for the full cost of the item.
  \item If the equipment is not returned properly and fully cleaned, at the
    discretion of the Executive Board. In the case of MathSoc clubs, who do not
   pay a deposit, a \$50 cleaning fee will be charged.
\end{enumerate}

The borrower agrees to be the sole operator of the equipment, or to be present
at all times during its use.

For organizations that are not affiliated with MathSoc, any equipment,
collectively, may be booked 5 times per term. (i.e. All pieces of equipment
share the same allowed use count for the term). This limit may be waived at the
discretion of the Vice President, Operations and one other member of the
Executive Board.

